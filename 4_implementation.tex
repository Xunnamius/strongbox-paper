\section{StrongBox Implementation}\label{sec:implementation}

Our implementation of StrongBox is comprised of 5000 lines of C code. StrongBox
uses OpenSSL version 1.0.2 and LibSodium version 1.0.12 for its ChaCha20,
Argon2, Blake2, and AES-XTS implementations, likewise implemented in C. The
SHA-256 Merkle Tree implementation is borrowed from the Secure Block Device
library~\cite{SBD}. StrongBox's implementation is available as
open-source.\footnote{\StrongBoxURI}

To reduce the complexity of the experimental setup, establish a fair baseline,
and allow StrongBox to run in user space, we use a BUSE~\cite{BUSE} virtual
block device. BUSE is a thin (200 LOC) wrapper around the standard Linux Network
Block Device (NBD), which allows a machine to serve requests for reads and
writes to virtual block devices exposed via domain socket. We built StrongBox on
top of BUSE/NBD because a simple block device in user space allows for quick
experimentation and rapid prototyping. It is not required for a proper
implementation.

\subsection{Deriving Subkeys}
The cryptographic driver requires a shared master secret. The
derivation of this master secret is implementation specific and has no
impact on performance as it is completed during StrongBox's
initialization. Our implementation uses the Argon2 KDF to derive a
master secret from a given password with an acceptable time-memory
trade-off.

% Describe deriving the nugget keys from the master secret
To assign each nugget its own unique keystream, that nugget requires a
unique key and associated nonce. We derive these nugget subkeys from
the master secret during StrongBox's initialization. To guarantee the
backing store's integrity, each flake is tagged with a MAC. We use
Poly1305, accepting a 32-byte one-time key and a plaintext of
arbitrary length to generate tags. These one-time flake subkeys are
derived from their respective nugget subkeys.

\subsection{A Secure, Persistent, Monotonic Counter} 
Our target platform uses an embedded Multi-Media Card (eMMC) as a
backing store. In addition to boot and user data partitions, the eMMC
standard includes a secure storage partition called a Replay Protected
Memory Block (RPMB)~\cite{eMMC-standard}. The RPMB partition's size
is configurable to be at most 16MB (32MB on some Samsung
devices)~\cite{RPMB}. All read and write commands issued to the RPMB
must be authenticated by a key burned into write-once storage
(typically eFUSE) during some one-time, secure initialization process.

To implement rollback protection on top of the RPMB, the key for
authenticating RPMB commands can be contained in TEE sealed storage or
derived from the TPM. For this implementation, StrongBox requires
interaction with TPM/TEE secure storage only at mount time, where the
authentication key can be retrieved and cached for the duration of
StrongBox's lifetime. With the cached key on hand, our implementation
makes traditional IOCTL calls to read and write global version counter
data to the RPMB eMMC partition, enforcing the invariant that it only
increase monotonically.

Our design is not dependent on the eMMC standard, however. Trusted
hardware mechanisms other than the eMMC RPMB partition, including
TPMs, support secure, persistent storage and/or monotonic counters
directly. These can be adapted for use with StrongBox just as well.

There are two practical concerns we must address while implementing
the secure counter: wear and performance overhead. Wear is a concern
because the counter is implemented in non-volatile storage. The RPMB
implements all the same wear protection mechanisms that are used to
store user-data~\cite{eMMC-standard}. Additionally, StrongBox writes
to the global version counter once per write to user-data. Given that
the eMMC implements the same wear protection for the RPMB and user
data, and that the ratio of writes to these areas is 1:1, we expect
StrongBox places no additional wear burden on the hardware. Further,
with the JEDEC spec suggesting RPMB implementations use more durable
and faster single-level NAND flash cells rather than cheaper and
slower multi-level NAND flash cells \cite{eMMC-standard,RPMB}, the
RPMB partition will likely outlive and outperform the user-data
portion of the eMMC.

In terms of performance overhead, updating the global version counter
requires making one 64-bit authenticated write per user-data write. As
user-data writes are almost always substantially larger, we see no
significant overhead from the using the RPMB to store the secure
counter.

\subsection{LFS Garbage Collection}

An LFS attempts to write to a drive sequentially in an append-only fashion, as
if writing to a log. This requires large amounts of contiguous space, called
\emph{segments}. Since any backing store is necessarily finite, an LFS can only
append so much data before it runs out of space. When this occurs, the LFS
triggers a \emph{segment cleaning algorithm} to erase outdated data and
compress the remainder of the log into as few segments as
possible~\cite{LFS,F2FS}. This procedure is known more broadly as \emph{garbage
collection}~\cite{F2FS}.

In the context of StrongBox, garbage collection could potentially incur high
overhead. The procedure itself would, with its every write, require a rekeying
of any affected nuggets. Worse, every proceeding write would appear to StrongBox
as if it were an overwrite, since there is no way for StrongBox to know that the
LFS triggered garbage collection internally.

In practice, modern production LFSes are optimized to perform garbage collection
as few times as possible~\cite{F2FS}. Further, they often perform garbage
collection in a background thread that triggers when the filesystem is idle and
only perform expensive on-demand garbage collection when the backing store is
nearing capacity~\cite{F2FS, NILFS}. We leave garbage collection turned on for
all of our tests and see no substantial performance degradation from this
process because it is scheduled not to interfere with user I/O.

\subsection{Overhead}

StrongBox stores metadata on the drive it is encrypting (see
\figref{backstore}). This metadata should be small compared to the user data.
Our implementation uses 4KB flakes, 256 flakes/nugget, and 1024 nuggets per GB
of user data. Given the flake and nugget overhead, this configuration requires
just over 40KB of metadata per 1 GB of user data. There is an additional, single
static header that requires just over 200 bytes. \emph{Thus StrongBox's overhead
in terms of storage is less than one hundredth of a percent.}

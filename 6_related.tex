\section{Related Work} \label{sec:related}

Some of the most popular cryptosystems offering a confidentiality
guarantee for data at rest employ a symmetric encryption scheme known
as a Tweakable Enciphering Scheme (TES)~\cite{STES,XEX}. There have
been numerous TES-based constructions securing data at
rest~\cite{STES,CMC,HCTR}, including the well known XEX-based XTS
operating mode of AES~\cite{XTS} explored earlier in this work. Almost
all TES constructions and the storage management systems that
implement them use one or more block ciphers as their primary
primitive~\cite{TES-From-Stream-Cipher,STES}.

Our StrongBox implementation borrows from the design of these systems. One in
particular is \emph{dm-crypt}, a Linux framework employing a
\textit{LinuxDeviceMapper} to provide a virtual block interface for physical
block devices. Dm-crypt provides an implementation of the AES-XTS algorithm among
others and is used widely in the Linux ecosystem~\cite{DmC-Android, dmcrypt}.
The algorithms provided by dm-crypt all employ block ciphers~\cite{dmcrypt}.
Instead of a block cipher, however, StrongBox uses a stream cipher to provide
the same confidentiality guarantee and consistent or better I/O performance.
Further unlike dm-crypt and other similar virtualization frameworks, StrongBox's
ciphering operations do not require sector level tweaks, depending on the
implementation. With StrongBox, several physical blocks consisting of one or
more sectors are considered as discrete logical units, \ie nuggets and flakes.

Substituting a block cipher for a stream cipher forms the core of several
contributions to the state-of-the-art~\cite{STES, TES-From-Stream-Cipher}.
Chakraborty et al. proposed STES---a stream cipher based low cost scheme for
securing stored data~\cite{STES}. STES is a novel TES which can be implemented
compactly with low overall power consumption. It combines a stream cipher and a
universal hash function via XOR and is targeting low cost FPGAs to provide
confidentiality of data on USBs and SD cards. Our StrongBox, on the other hand,
is not a TES and does not directly implement a TES. StrongBox combines a stream
cipher with nonce ``tweak'' and nugget data via XOR and is targeting any
configuration employing a well-behaved Log-structured Filesystem (LFS) at some
level to provide confidentiality of data.

Offering a transparent cryptographic layer at the block device level has been
proposed numerous times~\cite{SBD}. Production implementations include storage
management systems like dm-crypt. Specifically, Hein et al. proposed the Secure
Block Device (SBD)~\cite{SBD}---an ARM TrustZone secure world transparent
filesystem encryption layer optimized for ANDIX OS and implemented and evaluated
using the Linux Network Block Device (NBD) driver. StrongBox is also implemented
and evaluated using the NBD, but is not limited to one specific operating
system. Further unlike StrongBox, SBD is not explicitly designed for use outside
of the ARM TrustZone secure world. Contrarily, StrongBox was designed to be used
on any system that provides a subset of functionality provided by a Trusted
Platform Module (TPM) and/or Trusted Execution Environment (TEE). Specifically,
StrongBox requires the availability of a dedicated hardware protected secure
monotonic counter to prevent rollback attacks and ensure the freshness of
StrongBox. The primary design goal of StrongBox is to achieve provide higher
performance than the industry standard AES-XTS algorithm utilizing a stream
cipher.

StrongBox's design is only possible because of the availability of
hardware support for security, which has been a major thrust of
research efforts
~\cite{asplos1,asplos2,asplos3,asplos4,isca1,isca2},
and is now available in almost all commercial mobile processors
~\cite{TPM,TEE,RPMB,Kirovski}. Our implementation makes use of the
replay protected memory block on eMMC devices
~\cite{eMMC-standard,RPMB}, but it could be reimplemented using any
hardware that supports persistent, monotonic counters.

The combination of trusted hardware and monotonic counters enables new
security mechanisms. For example, van Dijk et al. use this
combination allow clients to securly store data on an untrusted server
~\cite{CSAIL-TPM}. Like StrongBox, their approach relies on trusted
hardware (TPM specifically \cite{TPM}), logs, and monotonic counters.
The van Dijk et al. approach, however, uses existing secure storage
and is not concerned with storage speed. StrongBox uses these same
mechanisms along with novel metadata layout and system design to solve
a different problem: providing higher performance than AES-XTS based
approaches.

Achieving on-drive data integrity protection through the use of checksums has
been used by filesystems and many other storage management systems. Examples
include ZFS~\cite{ZFS} and others~\cite{SBD}. For our implementation of
StrongBox, we used the Merkle Tree library offered by SBD to manage our
in-memory checksum verification. A proper implementation of StrongBox need not
use the SDB SHA-256 Merkle Tree library. It was chosen for convenience.

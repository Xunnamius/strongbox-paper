\section{Introduction}\label{sec:introduction}

Full-drive encryption (FDE)\footnote{The common term is full-\emph{disk}
encryption, but this work targets SSDs, so we use \emph{drive}.} is an essential
technique for protecting the privacy of data at rest. For mobile devices,
maintaining data privacy is especially important as these devices contain
sensitive personal and financial data yet are easily lost or stolen. The current
standard for securing data at rest is to use the AES cipher in XTS
mode~\shortcite{XTS, NISTXTS}. Unfortunately, employing AES-XTS increases
read/write latency by 3--5$\times$ compared to unencrypted storage.

It is well known that authenticated encryption using \emph{stream}
ciphers---such as ChaCha20~\cite{ChaCha20}---is faster than using AES (see
\figref{motivation}). Indeed, Google made the case for stream ciphers over AES,
switching HTTPS connections on Chrome for Android to use a stream cipher for
better performance~\cite{google-blog}. Stream ciphers are not used for FDE,
however, for two reasons: (1) confidentiality and (2) performance. First, when
applied naively to stored data, stream ciphers are trivially vulnerable to
attacks---including \emph{many-time pad and rollback attacks}---that reveal the
plaintext by overwriting a secure storage location with the same key. Second, it
has been assumed that adding the meta-data required to resist these attacks
would ruin the stream cipher's performance advantage. Thus, the conventional
wisdom is that FDE necessarily incurs the overhead of AES-XTS or a similar
primitive.

We argue that two technological shifts in mobile device hardware overturn this
conventional wisdom, enabling confidential, high-performance storage with stream
ciphers. First, these devices commonly employ solid-state storage with Flash
Translation Layers (FTL), which operate similarly to Log-structured File Systems
(LFS)~\cite{LFS,F2FS,NILFS}. Second, mobile devices now support trusted
hardware, such as Trusted Execution Environments (TEE)~\cite{TEE,TrustZone} and
secure storage areas~\cite{eMMC-standard}. FTLs and LFSes are used to limit
sector/cell overwrites, hence extending the life of the drive. Most writes
simply appended to a log, reducing the occurrence of overwrites and the chance
for attacks. The presence of secure hardware means that drive encryption modules
have access to persistent, monotonically increasing counters that can be used to
prevent rollback attacks when overwrites do occur.

Given these trends, we propose StrongBox, a new method for securing data at
rest. StrongBox is a drop-in replacement for AES-XTS-backed FDE such as
dm-crypt~\cite{dmcrypt}; \ie it requires no interface changes. The primary
challenge is that even with a FTL or LFS running above an SSD, filesystem blocks
will occasionally be overwritten; \eg by segment cleaning or \emph{garbage
collection}. StrongBox overcomes this challenge by using a fast stream cipher
for confidentiality and performance with integrity preserving Message
Authentication Codes~\cite{MAC} or ``MAC tags'' and a secure, persistent
hardware counter to ensure integrity and prevent attacks. \emph{StrongBox's main
contribution is a system design enabling the first confidential, high-
performance drive encryption based on a stream cipher.}

We demonstrate StrongBox's effectiveness on a mobile ARM big.LITTLE system---a
Samsung Exynos Octa 5---running Ubuntu Trusty 14.04 LTS, kernel 3.10.58. We use
ChaCha20~\cite{ChaCha20} as our stream cipher, Poly1305~\cite{Poly1305} as our
MAC algorithm, and the eMMC Replay Protected Memory Block partition to store a
secure counter \cite{eMMC-standard}. As StrongBox requires no change to any
existing interfaces, we benchmark it on two of the most popular LFSes:
NILFS~\cite{NILFS} and F2FS~\cite{F2FS}. We compare the performance of these
LFSes on top of AES-XTS (via dm-crypt) and StrongBox. Additionally, we compare
the performance of AES-XTS encrypted Ext4 filesystems with StrongBox and F2FS.
Our results show:

\begin{itemize}

  \item \emph{Improved read performance:} StrongBox provides decreased read
latencies across all tested filesystems in the majority of benchmarks when
compared to dm-crypt; \eg under F2FS, StrongBox provides as much as a
$2.36\times$ ($1.72\times$ average) speedup over AES-XTS\@.

  \item \emph{Equivalent write performance:} despite having to maintain more
metadata than FDE schemes based on AES-XTS, StrongBox achieves near parity or
provides an improvement in observed write latencies in the majority of
benchmarks; \eg under F2FS, StrongBox provides an average $1.27\times$ speedup
over AES-XTS\@.

\end{itemize}

StrongBox achieves these performance gains while providing a stronger integrity
guarantee than AES-XTS\@. Whereas XTS mode only hopes to randomize plaintext when
the ciphertext is altered~\cite{XTS}, StrongBox provides the security of
standard authenticated encryption. In addition, StrongBox's implementation is
available open-source.\footnote{\StrongBoxURI}
